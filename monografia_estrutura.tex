\documentclass[portuguese,12pt,oneside,a4paper]{book}

% ---------------------------------------------------------------------------- %
% Pacotes 
\usepackage[T1]{fontenc}
\usepackage[brazil]{babel}           % define a linguagem padrão
\usepackage[utf8x]{inputenc}
\usepackage[pdftex]{graphicx}           % usamos arquivos pdf/png como figuras
\usepackage{setspace}                   % espa�amento flex�vel
\usepackage{makeidx}                    % �ndice remissivo
\usepackage[nottoc]{tocbibind}          % acrescentamos a bibliografia/indice/conteudo no Table of Contents
\usepackage{courier}                    % usa o Adobe Courier no lugar de Computer Modern Typewriter
\usepackage{type1cm}                    % fontes realmente escal�veis
\usepackage{listings}                   % para formatar c�digo-fonte (ex. em Java)
\usepackage{titletoc}
\usepackage[fixlanguage]{babelbib}
\usepackage[font=small,format=plain,labelfont=bf,up,textfont=it,up]{caption}
\usepackage[usenames,svgnames,dvipsnames]{xcolor}
\usepackage[a4paper,top=2.54cm,bottom=2.0cm,left=2.0cm,right=2.54cm]{geometry} % margens
\usepackage[pdftex,plainpages=false,pdfpagelabels,pagebackref,colorlinks=true,citecolor=black,linkcolor=black,urlcolor=black,filecolor=black,bookmarksopen=true]{hyperref} % links em preto
\usepackage[all]{hypcap}                % soluciona o problema com o hyperref e capitulos

\fontsize{60}{62}\usefont{OT1}{cmr}{m}{n}{\selectfont}
\usepackage{color} % colorir texto
\usepackage{xcolor}
\usepackage{amsmath} %expressoes matematicas
\usepackage{pdflscape} %permite mudar orientacao da pagina para paisagem E MUDA NO PDF
\usepackage{longtable} %permite tabelas que ocupem mais de uma página
\usepackage{multicol}
\usepackage{enumitem} %para pode iniciar enumerate em 0
\usepackage{pdfpages}



% ---------------------------------------------------------------------------- %
% Cabe�alhos similares ao TAOCP de Donald E. Knuth
\usepackage{fancyhdr}
\pagestyle{fancy}
\fancyhf{}
\renewcommand{\chaptermark}[1]{\markboth{\MakeUppercase{#1}}{}}
\renewcommand{\sectionmark}[1]{\markright{\MakeUppercase{#1}}{}}
\renewcommand{\headrulewidth}{0pt}

% ---------------------------------------------------------------------------- %
\graphicspath{{./figuras/}}             % caminho das figuras (recomend�vel)
\frenchspacing                          % arruma o espa�o: id est (i.e.) e exempli gratia (e.g.) 
\urlstyle{same}                         % URL com o mesmo estilo do texto e n�o mono-spaced
\makeindex                              % para o �ndice remissivo
\raggedbottom                           % para n�o permitir espa�os extra no texto
\fontsize{60}{62}\usefont{OT1}{cmr}{m}{n}{\selectfont}
\normalsize

% ---------------------------------------------------------------------------- %
% Op��es de listing usados para o c�digo fonte
% Ref: http://en.wikibooks.org/wiki/LaTeX/Packages/Listings
\lstset{ %
language=Java,                  % choose the language of the code
basicstyle=\footnotesize,       % the size of the fonts that are used for the code
numbers=left,                   % where to put the line-numbers
numberstyle=\footnotesize,      % the size of the fonts that are used for the line-numbers
stepnumber=1,                   % the step between two line-numbers. If it's 1 each line will be numbered
numbersep=5pt,                  % how far the line-numbers are from the code
showspaces=false,               % show spaces adding particular underscores
showstringspaces=false,         % underline spaces within strings
showtabs=false,                 % show tabs within strings adding particular underscores
frame=single,	                % adds a frame around the code
framerule=0.6pt,
tabsize=2,	                    % sets default tabsize to 2 spaces
captionpos=b,                   % sets the caption-position to bottom
breaklines=true,                % sets automatic line breaking
breakatwhitespace=false,        % sets if automatic breaks should only happen at whitespace
escapeinside={\%*}{*)},         % if you want to add a comment within your code
backgroundcolor=\color[rgb]{1.0,1.0,1.0}, % choose the background color.
rulecolor=\color[rgb]{0.8,0.8,0.8},
extendedchars=true,
xleftmargin=10pt,
xrightmargin=10pt,
framexleftmargin=10pt,
framexrightmargin=10pt
}

\lstdefinelanguage{blah} % minha linguagem se chama blah
{
% Aqui vão os atributos dela. Para ver mais atributos, Google!
tabsize=2, % o tab possui 2 espaços
frame=shadowbox,% o tipo de box que usaremos
%rulesepcolor=\color{lightgrey}, %/ a cor pra fazer o efeito legal da box
captionpos=b,% o caption do código vai para baixo (botton) o default eh top
extendedchars=false, % eu usei isso para conseguir colocar expressoes matematicas utilizando o scapeinside tbm
escapeinside='', % usando aspas, pode-se usar qualquer comando dentro deles
morekeywords = {if, else, while, do, for, and, or, action, initialize} % palavras reservadas que ficaram em negrito
} 


% ABNT %%%%%%%%%%%%%%%%%%%%%%%%%%%%%%%%%%%%%%%%%%%%%%%%%%%%%%%%%%%%%%%%%%%%%%%%%%%%%
\usepackage[alf,% Para usar o estilo de citacao alfabetico (seção 9.2 em 6023/2000
abnt-etal-text=emph,% o et al. vai aparecer em italico
abnt-and-type=e,% separador de autores com ´e´
bibjustif,% bibliografia justificada no final
abnt-etal-cite=3]{abntex2cite} 
% No caso, a 6023 diz que se deve usar ‘et al.’ para mais de três autores, ou seja, a partir de quatro autores. Então, usarei abnt-etal-cite=3
\usepackage{cite}%%% NOVO

% ------------ colocar o [fulanclearo 2002] antes da descricao de referencia bibliografica
\makeatletter
\renewcommand\@biblabel[1]{[#1]~}% colocar citação na frente da lista de bibliografia
\renewcommand\citepunct{;~}%%% NOVO
%\renewcommand{\@biblabel}[1]{[\textbf{#1}]~}
\makeatother
\citebrackets[] % fazer com que as citacoes dentro do texto virem colchetes
% -------------------------------------------------------
%%%%%%%%%%%%%%%%%%%%%%%%%%%%%%%%%%%%%%%%%%%%%%%%%%%%%%%%%%%%%%%%%%%%%%%%%%%%%%%%%%%%


% ---------------------------------------------------------------------------- %
% Corpo do texto
\begin{document}
\frontmatter 
% cabe�alho para as p�ginas das se��es anteriores ao cap�tulo 1 (frontmatter)
\fancyhead[RO]{{\footnotesize\rightmark}\hspace{2em}\thepage}
\setcounter{tocdepth}{2}
\fancyhead[LE]{\thepage\hspace{2em}\footnotesize{\leftmark}}
\fancyhead[RE,LO]{}
\fancyhead[RO]{{\footnotesize\rightmark}\hspace{2em}\thepage}

\onehalfspacing  % espa�amento

% ---------------------------------------------------------------------------- %
% Capa

\thispagestyle{empty}
\begin{center}
    \includegraphics[scale=0.4]{ifsul} \\
    \Large Instituto Federal de Educação, Ciência e Tecnologia Sul-rio-grandense\\
    \textit{Campus} Sapucaia do Sul\\
    Coordenação do Curso Técnico em Informática\\

    \vspace*{4cm}

    \textbf{\LARGE{TITULO DO TRABALHO}}\\
    
    \vspace*{3.5cm}
    \Large{NOME DO ALUNO}
    
    \vskip 3cm

    Orientador: Nome do Orientador\\

    \vskip 3cm
    
    \normalsize{Sapucaia do Sul, fevereiro de 2020} \\
    \textregistered Nome do aluno
    
\end{center}

\newpage

\thispagestyle{empty}
\begin{center}
    \includegraphics[scale=0.4]{ifsul} \\
    \Large Instituto Federal de Educação, Ciência e Tecnologia Sul-rio-grandense\\
    \textit{Campus} Sapucaia do Sul\\
    Coordenação do Curso Técnico em Informática\\

    \vspace*{4cm}

    \textbf{\LARGE{Meu exemplo de TCC}}\\
    
    \vspace*{2cm}
    \Large{Felipe André Zeiser}
    
    \vskip 1.5cm
\end{center}    
    


    \begin{flushright} 
		\begin{minipage}[left]{0.5\linewidth}
	
			Monografia apresentada à Coordenação do Curso Técnico em Informática do IFSul - \textit{Campus} Sapucaia do Sul, como requisito parcial para conclusão do curso Técnico em Informática.\\                                                                
			
		\end{minipage}
                  
   	\end{flushright}                                


\begin{center}    

    \vskip 2cm
    Orientador: Fulano de Tal\\
    Co-orientador: Ciclano de Tal
    \vskip 1.0cm

    \normalsize{Sapucaia do Sul, Fevereiro de 2021}
\end{center}

\newpage

% ---------------------------------------------------------------------------- %
% FICHA CATALOGRÁFICA (VERSÃO FINAL/CORRIGDA)
% ---------------------------------------------------------------------------- %

%\includepdf{ficha-catalografica.pdf} %Incluir a ficha fornecida pela biblioteca do IFSul


% ---------------------------------------------------------------------------- %
% P�gina de rosto (s� para a vers�o final)
% ---------------------------------------------------------------------------- %

\newpage
\thispagestyle{empty}
    \begin{center}
        

	    \textbf{\LARGE{Meu exemplo de TCC}}\\

        \vspace*{3cm}

    	\textbf{FELIPE ANDRÉ ZEISER}
    
    	\vskip 3cm


		\rule{10cm}{0.1mm}
		XXXXXXXXXXXXXXXXXXXXXXXXXXXXX\\
		Orientador
		
    	\vskip 2cm
    			
		\rule{10cm}{0.1mm}
		XXXXXXXXXXXXXXXXXXXXXXXXXXXXXXX\\
		Membro da Banca
			
    	\vskip 2cm
		
		\rule{10cm}{0.1mm}
		XXXXXXXXXXXXXXXXXXXXXXXXXXXXXXXX\\
		Membro da Banca
			
		\vskip 3cm
			
		Sapucaia do Sul, Rio Grande do Sul, Brasil\\
		Fevereiro/2021

    \end{center}
\pagebreak

\newpage
\thispagestyle{empty}

        
        \vspace*{20.0cm}

        \begin{flushright}
			Aos meus familiares.
        \end{flushright}



\pagebreak

\newpage
\thispagestyle{empty}

        
        \vspace*{20.0cm}

        \begin{flushright}

             Alguma frase/pensamento (OPCIONAL)\\
             Autor\\
                   

        \end{flushright}



\pagebreak



 
% % ---------------------------------------------------------------------------- %
% % Agradecimentos

\chapter*{Agradecimentos}


	XXXXXXXXXXXXXXXXXXXXXX


\pagebreak


% ---------------------------------------------------------------------------- %
% Resumo

\chapter*{Resumo}
	XXXXXXXXXXXXXXXXXXXXXXXXXXXXXXXXXXXXXX.\\

\noindent \textbf{Palavras-chave:} XXXXXXXXXXXXXXXXXXX.


\pagebreak


% ---------------------------------------------------------------------------- %

 \chapter*{Abstract}
XXXXXXXXXXXXXXXXXXXXXXXXXXXXXXXXXXXXXXXXXXX.\\

 \noindent \textbf{Keywords:} XXXXXXXXXXXXXXXXXXXXXXXXXXXXXXXX.



% ---------------------------------------------------------------------------- %
% Sum�rio
\tableofcontents    % imprime o sum�rio


% ---------------------------------------------------------------------------- %
\chapter{Lista de Abreviaturas}
\begin{tabular}{ll}
	API 	& \textit{Application Programming Interface} \\
    BPL		& \textit{Bandpass Liftering} (Filtragem passa-faixa)\\
	CFG 	& \textit{Context-Free Grammar} (Gramática Livre de Contexto)\\
	CORDIC  & \textit{COordinate Rotation DIgital Computer} \\
	DARPA	& \textit{Defense Advanced Research Projects Agency}\\
	DTW 	& \textit{Dynamic Time Warping} (Alinhamento Dinâmico no Tempo)\\
	SBC 	& Sociedade Brasileira de Computação\\

\end{tabular}



% ---------------------------------------------------------------------------- %
% Listas de figuras e tabelas criadas automaticamente
\listoffigures            
\listoftables            


% ---------------------------------------------------------------------------- %
% Cap�tulos do trabalho
\mainmatter

% cabe�alho para as p�ginas de todos os cap�tulos
\fancyhead[RE,LO]{\thesection}

\onehalfspacing            % espa�amento um e meio

\input cap-introducao        % associado ao arquivo: 'cap-introducao.tex'
\input cap-modelagem
\input cap-finais

% ---------------------------------------------------------------------------- %
% Apêndices e/ou Anexos

\appendix
\renewcommand{\appendixname}{Apêndice} % Para trocar Apendice por Anexo (Se necessário)
\setcounter{chapter}{0} % ressetar a contagem

\chapter{Base de Dados}
\label{ape:base}

Capítulo opcional usado para que o autor coloque um texto,documento ou informação referente ao assunto do trabalho, mas que
não deve interromper a leitura. 




% ---------------------------------------------------------------------------- %
% Bibliografia
\backmatter \singlespacing   % espa�amento simples
\bibliographystyle{abntex2-alf}% cita��o bibliogr�fica alpha alpha-ime
\bibliography{bibliografia}  % associado ao arquivo: 'bibliografia.bib'


\end{document}
